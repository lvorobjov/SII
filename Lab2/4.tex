
Анализируется таблица от 1 до 41.
$$ 
\begin{array}{l|c|ccccccccc}
	  & C_{0} & C_{1} & C_{2} & C_{3} & C_{4} & C_{5} & C_{6} & C_{7} & C_{8} & C_{9}\\
 \textrm{GAIN} & 1.58 & 1.30 & 0.82 & 0.39 & 0.41 & 0.67 & 0.21 & 0.80 & 0.50 & 0.26
\end{array}
 $$
$$
T = \left( \begin{array}{lc|c|ccccccccc}
	 & G & C_{0} & C_{1} & C_{2} & C_{3} & C_{4} & C_{5} & C_{6} & C_{7} & C_{8} & C_{9}\\
	1 & 0 & 0 & 0 & 1 & 0 & 0 & 0 & 0 & 0 & 0 & 0\\
	2 & 0 & 0 & 1 & 1 & 0 & 0 & 0 & 0 & 0 & 0 & 0\\
	3 & 0 & 0 & 0 & 1 & 0 & 0 & 0 & 0 & 1 & 0 & 0\\
	4 & 0 & 0 & 0 & 1 & 0 & 0 & 0 & 0 & 0 & 1 & 0\\
	5 & 1 & 1 & 0 & 1 & 0 & 0 & 0 & 0 & 0 & 0 & 0\\
	6 & 1 & 2 & 3 & 0 & 0 & 0 & 0 & 0 & 4 & 0 & 0\\
	7 & 1 & 3 & 1 & 1 & 0 & 0 & 0 & 0 & 0 & 0 & 0\\
	8 & 4 & 4 & 2 & 0 & 0 & 0 & 0 & 0 & 1 & 0 & 0\\
	9 & 4 & 4 & 0 & 0 & 0 & 0 & 0 & 0 & 0 & 0 & 0\\
	10 & 1 & 4 & 2 & 0 & 0 & 0 & 1 & 1 & 0 & 0 & 0\\
	11 & 4 & 4 & 2 & 0 & 0 & 0 & 1 & 0 & 0 & 0 & 0\\
	12 & 0 & 5 & 0 & 0 & 0 & 0 & 0 & 0 & 0 & 0 & 0\\
	13 & 0 & 5 & 3 & 0 & 0 & 0 & 0 & 0 & 0 & 0 & 0\\
	14 & 6 & 5 & 0 & 0 & 0 & 0 & 0 & 0 & 0 & 1 & 0\\
	15 & 6 & 5 & 0 & 0 & 0 & 0 & 0 & 0 & 0 & 2 & 0\\
	16 & 2 & 5 & 0 & 0 & 1 & 1 & 0 & 0 & 0 & 0 & 0\\
	17 & 6 & 5 & 0 & 0 & 0 & 0 & 0 & 0 & 2 & 0 & 0\\
	18 & 2 & 5 & 0 & 0 & 1 & 1 & 0 & 0 & 0 & 0 & 0\\
	19 & 6 & 5 & 0 & 0 & 0 & 0 & 1 & 0 & 1 & 0 & 0\\
	20 & 5 & 5 & 2 & 0 & 0 & 0 & 0 & 0 & 0 & 0 & 2\\
	21 & 5 & 5 & 2 & 0 & 0 & 0 & 1 & 0 & 1 & 0 & 0\\
	22 & 3 & 5 & 0 & 0 & 1 & 1 & 1 & 0 & 0 & 0 & 0\\
	23 & 4 & 5 & 4 & 0 & 0 & 0 & 0 & 0 & 0 & 0 & 0\\
	24 & 4 & 5 & 2 & 0 & 0 & 0 & 0 & 0 & 0 & 2 & 0\\
	25 & 5 & 5 & 2 & 0 & 0 & 0 & 0 & 0 & 0 & 0 & 0\\
	26 & 4 & 5 & 1 & 0 & 1 & 1 & 1 & 0 & 0 & 0 & 0\\
	27 & 4 & 5 & 2 & 0 & 0 & 0 & 0 & 0 & 0 & 0 & 1\\
	28 & 4 & 5 & 2 & 0 & 0 & 0 & 0 & 0 & 0 & 1 & 0\\
	29 & 4 & 5 & 0 & 0 & 1 & 2 & 0 & 0 & 0 & 0 & 0\\
	30 & 4 & 5 & 0 & 0 & 0 & 0 & 0 & 0 & 3 & 0 & 0\\
	31 & 4 & 5 & 0 & 0 & 0 & 0 & 0 & 0 & 4 & 0 & 0\\
	32 & 4 & 5 & 0 & 0 & 0 & 0 & 0 & 0 & 0 & 0 & 1\\
	33 & 0 & 6 & 0 & 1 & 0 & 0 & 0 & 0 & 0 & 0 & 0\\
	34 & 1 & 7 & 0 & 1 & 0 & 0 & 1 & 1 & 0 & 0 & 0\\
	35 & 2 & 7 & 0 & 1 & 0 & 0 & 0 & 0 & 1 & 2 & 0\\
	36 & 3 & 7 & 2 & 0 & 0 & 0 & 0 & 0 & 1 & 0 & 0\\
	37 & 3 & 7 & 2 & 0 & 0 & 0 & 1 & 0 & 0 & 0 & 0\\
	38 & 2 & 7 & 0 & 0 & 0 & 0 & 0 & 0 & 0 & 0 & 0\\
	39 & 2 & 7 & 2 & 1 & 0 & 0 & 0 & 0 & 0 & 0 & 0\\
	40 & 2 & 7 & 0 & 1 & 0 & 0 & 1 & 0 & 0 & 0 & 0\\
	41 & 2 & 7 & 3 & 0 & 0 & 0 & 0 & 0 & 0 & 0 & 0\\
\end{array} \right)
$$

Найдено правило: $C_{0} = 0 \Longrightarrow G = 0$.

Найдено правило: $C_{0} = 1 \Longrightarrow G = 1$.

Найдено правило: $C_{0} = 2 \Longrightarrow G = 1$.

Найдено правило: $C_{0} = 3 \Longrightarrow G = 1$.

Анализируется таблица от 8 до 11.
$$ 
\begin{array}{lcccccc|c|ccc}
	  & C_{0} & C_{1} & C_{2} & C_{3} & C_{4} & C_{5} & C_{6} & C_{7} & C_{8} & C_{9}\\
 \textrm{GAIN} & 0.00 & -0.06 & 0.00 & 0.00 & 0.00 & 0.06 & 0.45 & -0.06 & 0.00 & 0.00
\end{array}
 $$
$$
T = \left( \begin{array}{lccccccc|c|ccc}
	 & G & C_{0} & C_{1} & C_{2} & C_{3} & C_{4} & C_{5} & C_{6} & C_{7} & C_{8} & C_{9}\\
	8 & 4 & 4 & 2 & 0 & 0 & 0 & 0 & 0 & 1 & 0 & 0\\
	9 & 4 & 4 & 0 & 0 & 0 & 0 & 0 & 0 & 0 & 0 & 0\\
	10 & 4 & 4 & 2 & 0 & 0 & 0 & 1 & 0 & 0 & 0 & 0\\
	11 & 1 & 4 & 2 & 0 & 0 & 0 & 1 & 1 & 0 & 0 & 0\\
\end{array} \right)
$$

Найдено правило: $C_{6} = 0 \Longrightarrow G = 4$.

Найдено правило: $C_{6} = 1 \Longrightarrow G = 1$.

Завершен анализ таблицы от 8 до 11.

Анализируется таблица от 12 до 32.
$$ 
\begin{array}{lc|c|cccccccc}
	  & C_{0} & C_{1} & C_{2} & C_{3} & C_{4} & C_{5} & C_{6} & C_{7} & C_{8} & C_{9}\\
 \textrm{GAIN} & 0.00 & 1.03 & 0.00 & 0.73 & 0.78 & 0.54 & 0.00 & 0.62 & 0.45 & 0.34
\end{array}
 $$
$$
T = \left( \begin{array}{lcc|c|cccccccc}
	 & G & C_{0} & C_{1} & C_{2} & C_{3} & C_{4} & C_{5} & C_{6} & C_{7} & C_{8} & C_{9}\\
	12 & 0 & 5 & 0 & 0 & 0 & 0 & 0 & 0 & 0 & 0 & 0\\
	13 & 4 & 5 & 0 & 0 & 0 & 0 & 0 & 0 & 0 & 0 & 1\\
	14 & 6 & 5 & 0 & 0 & 0 & 0 & 0 & 0 & 0 & 1 & 0\\
	15 & 6 & 5 & 0 & 0 & 0 & 0 & 0 & 0 & 0 & 2 & 0\\
	16 & 2 & 5 & 0 & 0 & 1 & 1 & 0 & 0 & 0 & 0 & 0\\
	17 & 6 & 5 & 0 & 0 & 0 & 0 & 0 & 0 & 2 & 0 & 0\\
	18 & 2 & 5 & 0 & 0 & 1 & 1 & 0 & 0 & 0 & 0 & 0\\
	19 & 6 & 5 & 0 & 0 & 0 & 0 & 1 & 0 & 1 & 0 & 0\\
	20 & 4 & 5 & 0 & 0 & 0 & 0 & 0 & 0 & 3 & 0 & 0\\
	21 & 4 & 5 & 0 & 0 & 1 & 2 & 0 & 0 & 0 & 0 & 0\\
	22 & 3 & 5 & 0 & 0 & 1 & 1 & 1 & 0 & 0 & 0 & 0\\
	23 & 4 & 5 & 0 & 0 & 0 & 0 & 0 & 0 & 4 & 0 & 0\\
	24 & 4 & 5 & 1 & 0 & 1 & 1 & 1 & 0 & 0 & 0 & 0\\
	25 & 5 & 5 & 2 & 0 & 0 & 0 & 0 & 0 & 0 & 0 & 0\\
	26 & 4 & 5 & 2 & 0 & 0 & 0 & 0 & 0 & 0 & 2 & 0\\
	27 & 4 & 5 & 2 & 0 & 0 & 0 & 0 & 0 & 0 & 0 & 1\\
	28 & 4 & 5 & 2 & 0 & 0 & 0 & 0 & 0 & 0 & 1 & 0\\
	29 & 5 & 5 & 2 & 0 & 0 & 0 & 1 & 0 & 1 & 0 & 0\\
	30 & 5 & 5 & 2 & 0 & 0 & 0 & 0 & 0 & 0 & 0 & 2\\
	31 & 0 & 5 & 3 & 0 & 0 & 0 & 0 & 0 & 0 & 0 & 0\\
	32 & 4 & 5 & 4 & 0 & 0 & 0 & 0 & 0 & 0 & 0 & 0\\
\end{array} \right)
$$

Анализируется таблица от 12 до 23.
$$ 
\begin{array}{lcccc|c|ccccc}
	  & C_{0} & C_{1} & C_{2} & C_{3} & C_{4} & C_{5} & C_{6} & C_{7} & C_{8} & C_{9}\\
 \textrm{GAIN} & 0.00 & 0.00 & 0.00 & 0.86 & 0.99 & 0.52 & 0.00 & 0.73 & 0.38 & 0.17
\end{array}
 $$
$$
T = \left( \begin{array}{lccccc|c|ccccc}
	 & G & C_{0} & C_{1} & C_{2} & C_{3} & C_{4} & C_{5} & C_{6} & C_{7} & C_{8} & C_{9}\\
	12 & 0 & 5 & 0 & 0 & 0 & 0 & 0 & 0 & 0 & 0 & 0\\
	13 & 4 & 5 & 0 & 0 & 0 & 0 & 0 & 0 & 0 & 0 & 1\\
	14 & 6 & 5 & 0 & 0 & 0 & 0 & 0 & 0 & 0 & 1 & 0\\
	15 & 6 & 5 & 0 & 0 & 0 & 0 & 0 & 0 & 0 & 2 & 0\\
	16 & 4 & 5 & 0 & 0 & 0 & 0 & 0 & 0 & 4 & 0 & 0\\
	17 & 6 & 5 & 0 & 0 & 0 & 0 & 0 & 0 & 2 & 0 & 0\\
	18 & 4 & 5 & 0 & 0 & 0 & 0 & 0 & 0 & 3 & 0 & 0\\
	19 & 6 & 5 & 0 & 0 & 0 & 0 & 1 & 0 & 1 & 0 & 0\\
	20 & 2 & 5 & 0 & 0 & 1 & 1 & 0 & 0 & 0 & 0 & 0\\
	21 & 2 & 5 & 0 & 0 & 1 & 1 & 0 & 0 & 0 & 0 & 0\\
	22 & 3 & 5 & 0 & 0 & 1 & 1 & 1 & 0 & 0 & 0 & 0\\
	23 & 4 & 5 & 0 & 0 & 1 & 2 & 0 & 0 & 0 & 0 & 0\\
\end{array} \right)
$$

Анализируется таблица от 12 до 19.
$$ 
\begin{array}{lccccccc|c|cc}
	  & C_{0} & C_{1} & C_{2} & C_{3} & C_{4} & C_{5} & C_{6} & C_{7} & C_{8} & C_{9}\\
 \textrm{GAIN} & 0.00 & 0.00 & 0.00 & 0.00 & 0.00 & 0.10 & 0.00 & 0.59 & 0.26 & 0.16
\end{array}
 $$
$$
T = \left( \begin{array}{lcccccccc|c|cc}
	 & G & C_{0} & C_{1} & C_{2} & C_{3} & C_{4} & C_{5} & C_{6} & C_{7} & C_{8} & C_{9}\\
	12 & 0 & 5 & 0 & 0 & 0 & 0 & 0 & 0 & 0 & 0 & 0\\
	13 & 4 & 5 & 0 & 0 & 0 & 0 & 0 & 0 & 0 & 0 & 1\\
	14 & 6 & 5 & 0 & 0 & 0 & 0 & 0 & 0 & 0 & 1 & 0\\
	15 & 6 & 5 & 0 & 0 & 0 & 0 & 0 & 0 & 0 & 2 & 0\\
	16 & 6 & 5 & 0 & 0 & 0 & 0 & 1 & 0 & 1 & 0 & 0\\
	17 & 6 & 5 & 0 & 0 & 0 & 0 & 0 & 0 & 2 & 0 & 0\\
	18 & 4 & 5 & 0 & 0 & 0 & 0 & 0 & 0 & 3 & 0 & 0\\
	19 & 4 & 5 & 0 & 0 & 0 & 0 & 0 & 0 & 4 & 0 & 0\\
\end{array} \right)
$$

Анализируется таблица от 12 до 15.
$$ 
\begin{array}{lcccccccc|c|c}
	  & C_{0} & C_{1} & C_{2} & C_{3} & C_{4} & C_{5} & C_{6} & C_{7} & C_{8} & C_{9}\\
 \textrm{GAIN} & 0.00 & 0.00 & 0.00 & 0.00 & 0.00 & 0.00 & 0.00 & 0.00 & 0.75 & 0.63
\end{array}
 $$
$$
T = \left( \begin{array}{lccccccccc|c|c}
	 & G & C_{0} & C_{1} & C_{2} & C_{3} & C_{4} & C_{5} & C_{6} & C_{7} & C_{8} & C_{9}\\
	12 & 0 & 5 & 0 & 0 & 0 & 0 & 0 & 0 & 0 & 0 & 0\\
	13 & 4 & 5 & 0 & 0 & 0 & 0 & 0 & 0 & 0 & 0 & 1\\
	14 & 6 & 5 & 0 & 0 & 0 & 0 & 0 & 0 & 0 & 1 & 0\\
	15 & 6 & 5 & 0 & 0 & 0 & 0 & 0 & 0 & 0 & 2 & 0\\
\end{array} \right)
$$

Анализируется таблица от 12 до 13.
$$ 
\begin{array}{lccccccccc|c|}
	  & C_{0} & C_{1} & C_{2} & C_{3} & C_{4} & C_{5} & C_{6} & C_{7} & C_{8} & C_{9}\\
 \textrm{GAIN} & 0.00 & 0.00 & 0.00 & 0.00 & 0.00 & 0.00 & 0.00 & 0.00 & 0.00 & 0.50
\end{array}
 $$
$$
T = \left( \begin{array}{lcccccccccc|c|}
	 & G & C_{0} & C_{1} & C_{2} & C_{3} & C_{4} & C_{5} & C_{6} & C_{7} & C_{8} & C_{9}\\
	12 & 0 & 5 & 0 & 0 & 0 & 0 & 0 & 0 & 0 & 0 & 0\\
	13 & 4 & 5 & 0 & 0 & 0 & 0 & 0 & 0 & 0 & 0 & 1\\
\end{array} \right)
$$

Найдено правило: $C_{9} = 0 \Longrightarrow G = 0$.

Найдено правило: $C_{9} = 1 \Longrightarrow G = 4$.

Завершен анализ таблицы от 12 до 13.

Найдено правило: $C_{8} = 1 \Longrightarrow G = 6$.

Найдено правило: $C_{8} = 2 \Longrightarrow G = 6$.

Завершен анализ таблицы от 12 до 15.

Найдено правило: $C_{7} = 1 \Longrightarrow G = 6$.

Найдено правило: $C_{7} = 2 \Longrightarrow G = 6$.

Найдено правило: $C_{7} = 3 \Longrightarrow G = 4$.

Найдено правило: $C_{7} = 4 \Longrightarrow G = 4$.

Завершен анализ таблицы от 12 до 19.

Анализируется таблица от 20 до 22.
$$ 
\begin{array}{lccccc|c|cccc}
	  & C_{0} & C_{1} & C_{2} & C_{3} & C_{4} & C_{5} & C_{6} & C_{7} & C_{8} & C_{9}\\
 \textrm{GAIN} & 0.00 & 0.00 & 0.00 & 0.00 & 0.00 & 0.48 & 0.00 & 0.00 & 0.00 & 0.00
\end{array}
 $$
$$
T = \left( \begin{array}{lcccccc|c|cccc}
	 & G & C_{0} & C_{1} & C_{2} & C_{3} & C_{4} & C_{5} & C_{6} & C_{7} & C_{8} & C_{9}\\
	20 & 2 & 5 & 0 & 0 & 1 & 1 & 0 & 0 & 0 & 0 & 0\\
	21 & 2 & 5 & 0 & 0 & 1 & 1 & 0 & 0 & 0 & 0 & 0\\
	22 & 3 & 5 & 0 & 0 & 1 & 1 & 1 & 0 & 0 & 0 & 0\\
\end{array} \right)
$$

Найдено правило: $C_{5} = 0 \Longrightarrow G = 2$.

Найдено правило: $C_{5} = 1 \Longrightarrow G = 3$.

Завершен анализ таблицы от 20 до 22.

Найдено правило: $C_{4} = 2 \Longrightarrow G = 4$.

Завершен анализ таблицы от 12 до 23.

Найдено правило: $C_{1} = 1 \Longrightarrow G = 4$.

Анализируется таблица от 25 до 30.
$$ 
\begin{array}{lcccccccc|c|c}
	  & C_{0} & C_{1} & C_{2} & C_{3} & C_{4} & C_{5} & C_{6} & C_{7} & C_{8} & C_{9}\\
 \textrm{GAIN} & 0.00 & 0.00 & 0.00 & 0.00 & 0.00 & 0.07 & 0.00 & 0.07 & 0.24 & 0.15
\end{array}
 $$
$$
T = \left( \begin{array}{lccccccccc|c|c}
	 & G & C_{0} & C_{1} & C_{2} & C_{3} & C_{4} & C_{5} & C_{6} & C_{7} & C_{8} & C_{9}\\
	25 & 5 & 5 & 2 & 0 & 0 & 0 & 0 & 0 & 0 & 0 & 0\\
	26 & 5 & 5 & 2 & 0 & 0 & 0 & 0 & 0 & 0 & 0 & 2\\
	27 & 4 & 5 & 2 & 0 & 0 & 0 & 0 & 0 & 0 & 0 & 1\\
	28 & 5 & 5 & 2 & 0 & 0 & 0 & 1 & 0 & 1 & 0 & 0\\
	29 & 4 & 5 & 2 & 0 & 0 & 0 & 0 & 0 & 0 & 1 & 0\\
	30 & 4 & 5 & 2 & 0 & 0 & 0 & 0 & 0 & 0 & 2 & 0\\
\end{array} \right)
$$

Анализируется таблица от 25 до 28.
$$ 
\begin{array}{lccccccccc|c|}
	  & C_{0} & C_{1} & C_{2} & C_{3} & C_{4} & C_{5} & C_{6} & C_{7} & C_{8} & C_{9}\\
 \textrm{GAIN} & 0.00 & 0.00 & 0.00 & 0.00 & 0.00 & -0.06 & 0.00 & -0.06 & 0.00 & 0.31
\end{array}
 $$
$$
T = \left( \begin{array}{lcccccccccc|c|}
	 & G & C_{0} & C_{1} & C_{2} & C_{3} & C_{4} & C_{5} & C_{6} & C_{7} & C_{8} & C_{9}\\
	25 & 5 & 5 & 2 & 0 & 0 & 0 & 0 & 0 & 0 & 0 & 0\\
	26 & 5 & 5 & 2 & 0 & 0 & 0 & 1 & 0 & 1 & 0 & 0\\
	27 & 4 & 5 & 2 & 0 & 0 & 0 & 0 & 0 & 0 & 0 & 1\\
	28 & 5 & 5 & 2 & 0 & 0 & 0 & 0 & 0 & 0 & 0 & 2\\
\end{array} \right)
$$

Найдено правило: $C_{9} = 0 \Longrightarrow G = 5$.

Найдено правило: $C_{9} = 1 \Longrightarrow G = 4$.

Найдено правило: $C_{9} = 2 \Longrightarrow G = 5$.

Завершен анализ таблицы от 25 до 28.

Найдено правило: $C_{8} = 1 \Longrightarrow G = 4$.

Найдено правило: $C_{8} = 2 \Longrightarrow G = 4$.

Завершен анализ таблицы от 25 до 30.

Найдено правило: $C_{1} = 3 \Longrightarrow G = 0$.

Найдено правило: $C_{1} = 4 \Longrightarrow G = 4$.

Завершен анализ таблицы от 12 до 32.

Найдено правило: $C_{0} = 6 \Longrightarrow G = 0$.

Анализируется таблица от 34 до 41.
$$ 
\begin{array}{lc|c|cccccccc}
	  & C_{0} & C_{1} & C_{2} & C_{3} & C_{4} & C_{5} & C_{6} & C_{7} & C_{8} & C_{9}\\
 \textrm{GAIN} & 0.00 & 0.47 & 0.35 & 0.00 & 0.00 & 0.33 & 0.44 & 0.17 & 0.05 & 0.00
\end{array}
 $$
$$
T = \left( \begin{array}{lcc|c|cccccccc}
	 & G & C_{0} & C_{1} & C_{2} & C_{3} & C_{4} & C_{5} & C_{6} & C_{7} & C_{8} & C_{9}\\
	34 & 1 & 7 & 0 & 1 & 0 & 0 & 1 & 1 & 0 & 0 & 0\\
	35 & 2 & 7 & 0 & 1 & 0 & 0 & 0 & 0 & 1 & 2 & 0\\
	36 & 2 & 7 & 0 & 1 & 0 & 0 & 1 & 0 & 0 & 0 & 0\\
	37 & 2 & 7 & 0 & 0 & 0 & 0 & 0 & 0 & 0 & 0 & 0\\
	38 & 3 & 7 & 2 & 0 & 0 & 0 & 1 & 0 & 0 & 0 & 0\\
	39 & 2 & 7 & 2 & 1 & 0 & 0 & 0 & 0 & 0 & 0 & 0\\
	40 & 3 & 7 & 2 & 0 & 0 & 0 & 0 & 0 & 1 & 0 & 0\\
	41 & 2 & 7 & 3 & 0 & 0 & 0 & 0 & 0 & 0 & 0 & 0\\
\end{array} \right)
$$

Анализируется таблица от 34 до 37.
$$ 
\begin{array}{lcccccc|c|ccc}
	  & C_{0} & C_{1} & C_{2} & C_{3} & C_{4} & C_{5} & C_{6} & C_{7} & C_{8} & C_{9}\\
 \textrm{GAIN} & 0.00 & 0.00 & -0.06 & 0.00 & 0.00 & 0.06 & 0.45 & -0.06 & -0.06 & 0.00
\end{array}
 $$
$$
T = \left( \begin{array}{lccccccc|c|ccc}
	 & G & C_{0} & C_{1} & C_{2} & C_{3} & C_{4} & C_{5} & C_{6} & C_{7} & C_{8} & C_{9}\\
	34 & 2 & 7 & 0 & 0 & 0 & 0 & 0 & 0 & 0 & 0 & 0\\
	35 & 2 & 7 & 0 & 1 & 0 & 0 & 0 & 0 & 1 & 2 & 0\\
	36 & 2 & 7 & 0 & 1 & 0 & 0 & 1 & 0 & 0 & 0 & 0\\
	37 & 1 & 7 & 0 & 1 & 0 & 0 & 1 & 1 & 0 & 0 & 0\\
\end{array} \right)
$$

Найдено правило: $C_{6} = 0 \Longrightarrow G = 2$.

Найдено правило: $C_{6} = 1 \Longrightarrow G = 1$.

Завершен анализ таблицы от 34 до 37.

Анализируется таблица от 38 до 40.
$$ 
\begin{array}{lcc|c|ccccccc}
	  & C_{0} & C_{1} & C_{2} & C_{3} & C_{4} & C_{5} & C_{6} & C_{7} & C_{8} & C_{9}\\
 \textrm{GAIN} & 0.00 & 0.00 & 0.48 & 0.00 & 0.00 & 0.04 & 0.00 & 0.04 & 0.00 & 0.00
\end{array}
 $$
$$
T = \left( \begin{array}{lccc|c|ccccccc}
	 & G & C_{0} & C_{1} & C_{2} & C_{3} & C_{4} & C_{5} & C_{6} & C_{7} & C_{8} & C_{9}\\
	38 & 3 & 7 & 2 & 0 & 0 & 0 & 1 & 0 & 0 & 0 & 0\\
	39 & 3 & 7 & 2 & 0 & 0 & 0 & 0 & 0 & 1 & 0 & 0\\
	40 & 2 & 7 & 2 & 1 & 0 & 0 & 0 & 0 & 0 & 0 & 0\\
\end{array} \right)
$$

Найдено правило: $C_{2} = 0 \Longrightarrow G = 3$.

Найдено правило: $C_{2} = 1 \Longrightarrow G = 2$.

Завершен анализ таблицы от 38 до 40.

Найдено правило: $C_{1} = 3 \Longrightarrow G = 2$.

Завершен анализ таблицы от 34 до 41.

Завершен анализ таблицы от 1 до 41.
% require tikz-qtree
\begin{figure}[H]
	\sffamily
	\small
	\centering
	\begin{tikzpicture}
	\tikzset{level distance=60pt}
	\tikzset{every leaf node/.style={text width=2cm,align=center,anchor=north}}
	\tikzset{every internal node/.style={text width=4cm,align=center,anchor=north}}
	\Tree [.{Группа знака?} [.{Форма} \edge node[auto]{Треуг}; {\hspace{0pt}Предупреждаю...} \edge node[auto]{Треуг}; {\hspace{0pt}Знаки приори...} \edge node[auto]{Восьм}; {\hspace{0pt}Знаки приори...} \edge node[auto]{Ромб}; {\hspace{0pt}Знаки приори...} \edge node[auto]{Квара}; [.{Наличие красной стрелки} \edge node[auto]{Нет}; {\hspace{0pt}Информационн...} \edge node[auto]{Есть}; {\hspace{0pt}Знаки приори...} ] \edge node[auto]{Прямо}; {см. далее} \edge node[auto]{Буква}; {\hspace{0pt}Предупреждаю...} \edge node[auto]{Круг}; {см. далее} ]  ]
	\end{tikzpicture}
	\caption{Дерево утверждений и фактов (часть~1)}
\end{figure}
% require tikz-qtree
\begin{figure}[H]
	\sffamily
	\small
	\centering
	\begin{tikzpicture}
	\tikzset{level distance=60pt}
	\tikzset{every leaf node/.style={text width=2cm,align=center,anchor=north}}
	\tikzset{every internal node/.style={text width=4cm,align=center,anchor=north}}
	\Tree [.{Круг} [.{Цвет фона} \edge node[auto]{Белый}; [.{Наличие красной стрелки} \edge node[auto]{Нет}; {\hspace{0pt}Запрещающие ...} \edge node[auto]{Есть}; {\hspace{0pt}Знаки приори...} ] \edge node[auto]{Синий}; [.{Наличие окантовки} \edge node[auto]{Нет}; {\hspace{0pt}Предписывающ...} \edge node[auto]{Есть}; {\hspace{0pt}Запрещающие ...} ] \edge node[auto]{Красн}; {\hspace{0pt}Запрещающие ...} ]  ]
	\end{tikzpicture}
	\caption{Дерево утверждений и фактов (часть~2)}
\end{figure}
% require tikz-qtree
\begin{figure}[H]
	\sffamily
	\small
	\centering
	\begin{tikzpicture}
	\tikzset{level distance=60pt}
	\tikzset{every leaf node/.style={text width=2cm,align=center,anchor=north}}
	\tikzset{every internal node/.style={text width=4cm,align=center,anchor=north}}
	\Tree [.{Прямоугольник} [.{Цвет фона} \edge node[auto]{Белый}; {см. далее} \edge node[auto]{Желты}; {\hspace{0pt}Информационн...} \edge node[auto]{Синий}; {см. далее} \edge node[auto]{Красн}; {\hspace{0pt}Предупреждаю...} \edge node[auto]{Зелен}; {\hspace{0pt}Информационн...} ]  ]
	\end{tikzpicture}
	\caption{Дерево утверждений и фактов (часть~3)}
\end{figure}
% require tikz-qtree
\begin{figure}[H]
	\sffamily
	\small
	\centering
	\begin{tikzpicture}
	\tikzset{level distance=60pt}
	\tikzset{every leaf node/.style={text width=2cm,align=center,anchor=north}}
	\tikzset{every internal node/.style={text width=4cm,align=center,anchor=north}}
	\Tree [.{Синий} [.{Вид транспорта} \edge node[auto]{Отсут}; [.{Тип зданий} \edge node[auto]{Отсут}; {\hspace{0pt}Знаки сервис...} \edge node[auto]{Город}; {\hspace{0pt}Информационн...} \edge node[auto]{Загор}; {\hspace{0pt}Знаки сервис...} ] \edge node[auto]{Общес}; {\hspace{0pt}Информационн...} \edge node[auto]{Частн}; {\hspace{0pt}Информационн...} ]  ]
	\end{tikzpicture}
	\caption{Дерево утверждений и фактов (часть~4)}
\end{figure}
% require tikz-qtree
\begin{figure}[H]
	\sffamily
	\small
	\centering
	\begin{tikzpicture}
	\tikzset{level distance=60pt}
	\tikzset{every leaf node/.style={text width=2cm,align=center,anchor=north}}
	\tikzset{every internal node/.style={text width=4cm,align=center,anchor=north}}
	\Tree [.{Белый} [.{Форма внутреннего знака} \edge node[auto]{Нет з}; {см. далее} \edge node[auto]{Круг}; [.{Наличие стрелок} \edge node[auto]{Нет}; {\hspace{0pt}Запрещающие ...} \edge node[auto]{Есть}; {\hspace{0pt}Предписывающ...} ] \edge node[auto]{Квадр}; {\hspace{0pt}Информационн...} ]  ]
	\end{tikzpicture}
	\caption{Дерево утверждений и фактов (часть~5)}
\end{figure}
% require tikz-qtree
\begin{figure}[H]
	\sffamily
	\small
	\centering
	\begin{tikzpicture}
	\tikzset{level distance=60pt}
	\tikzset{every leaf node/.style={text width=2cm,align=center,anchor=north}}
	\tikzset{every internal node/.style={text width=4cm,align=center,anchor=north}}
	\Tree [.{Нет знака} [.{Характер текста} \edge node[auto]{Отсут}; {см. далее} \edge node[auto]{Числа}; {\hspace{0pt}Таблички к д...} \edge node[auto]{Дата,}; {\hspace{0pt}Таблички к д...} \edge node[auto]{Геора}; {\hspace{0pt}Информационн...} \edge node[auto]{Слово}; {\hspace{0pt}Информационн...} ]  ]
	\end{tikzpicture}
	\caption{Дерево утверждений и фактов (часть~6)}
\end{figure}
% require tikz-qtree
\begin{figure}[H]
	\sffamily
	\small
	\centering
	\begin{tikzpicture}
	\tikzset{level distance=60pt}
	\tikzset{every leaf node/.style={text width=2cm,align=center,anchor=north}}
	\tikzset{every internal node/.style={text width=4cm,align=center,anchor=north}}
	\Tree [.{Отсутствует} [.{Вид транспорта} \edge node[auto]{Отсут}; [.{Тип зданий} \edge node[auto]{Отсут}; {\hspace{0pt}Предупреждаю...} \edge node[auto]{Город}; {\hspace{0pt}Информационн...} ] \edge node[auto]{Общес}; {\hspace{0pt}Таблички к д...} \edge node[auto]{Частн}; {\hspace{0pt}Таблички к д...} ]  ]
	\end{tikzpicture}
	\caption{Дерево утверждений и фактов (часть~7)}
\end{figure}

\section{Продукционные правила}

\begin{enumerate}[itemindent=0pt]
	\item \begin{tabbing}
	\hspace{4em}\=\kill
	\bf ЕСЛИ \> \tabfill{Форма = Треугольник, вершина вверх} \\
	\bf ТО \> \tabfill{Группа = Предупреждающие знаки}
	\end{tabbing}
	\item \begin{tabbing}
	\hspace{4em}\=\kill
	\bf ЕСЛИ \> \tabfill{Форма = Треугольник, вершина вниз} \\
	\bf ТО \> \tabfill{Группа = Знаки приоритета}
	\end{tabbing}
	\item \begin{tabbing}
	\hspace{4em}\=\kill
	\bf ЕСЛИ \> \tabfill{Форма = Восьмиугольник} \\
	\bf ТО \> \tabfill{Группа = Знаки приоритета}
	\end{tabbing}
	\item \begin{tabbing}
	\hspace{4em}\=\kill
	\bf ЕСЛИ \> \tabfill{Форма = Ромб} \\
	\bf ТО \> \tabfill{Группа = Знаки приоритета}
	\end{tabbing}
	\item \begin{tabbing}
	\hspace{4em}\=\kill
	\bf ЕСЛИ \> \tabfill{Форма = Кварат} \\
	\bf И \> \tabfill{Наличие красной стрелки = Нет} \\
	\bf ТО \> \tabfill{Группа = Информационно-указательные знаки}
	\end{tabbing}
	\item \begin{tabbing}
	\hspace{4em}\=\kill
	\bf ЕСЛИ \> \tabfill{Форма = Кварат} \\
	\bf И \> \tabfill{Наличие красной стрелки = Есть} \\
	\bf ТО \> \tabfill{Группа = Знаки приоритета}
	\end{tabbing}
	\item \begin{tabbing}
	\hspace{4em}\=\kill
	\bf ЕСЛИ \> \tabfill{Форма = Прямоугольник} \\
	\bf И \> \tabfill{Цвет фона = Белый} \\
	\bf И \> \tabfill{Форма внутреннего знака = Нет знака} \\
	\bf И \> \tabfill{Характер текста = Отсутствует} \\
	\bf И \> \tabfill{Вид транспорта = Отсутствует} \\
	\bf И \> \tabfill{Тип зданий = Отсутствуют} \\
	\bf ТО \> \tabfill{Группа = Предупреждающие знаки}
	\end{tabbing}
	\item \begin{tabbing}
	\hspace{4em}\=\kill
	\bf ЕСЛИ \> \tabfill{Форма = Прямоугольник} \\
	\bf И \> \tabfill{Цвет фона = Белый} \\
	\bf И \> \tabfill{Форма внутреннего знака = Нет знака} \\
	\bf И \> \tabfill{Характер текста = Отсутствует} \\
	\bf И \> \tabfill{Вид транспорта = Отсутствует} \\
	\bf И \> \tabfill{Тип зданий = Городские} \\
	\bf ТО \> \tabfill{Группа = Информационно-указательные знаки}
	\end{tabbing}
	\item \begin{tabbing}
	\hspace{4em}\=\kill
	\bf ЕСЛИ \> \tabfill{Форма = Прямоугольник} \\
	\bf И \> \tabfill{Цвет фона = Белый} \\
	\bf И \> \tabfill{Форма внутреннего знака = Нет знака} \\
	\bf И \> \tabfill{Характер текста = Отсутствует} \\
	\bf И \> \tabfill{Вид транспорта = Общественный (автобус, троллейбус, трамвай, самолет, поезд)} \\
	\bf ТО \> \tabfill{Группа = Таблички к дорожным знакам}
	\end{tabbing}
	\item \begin{tabbing}
	\hspace{4em}\=\kill
	\bf ЕСЛИ \> \tabfill{Форма = Прямоугольник} \\
	\bf И \> \tabfill{Цвет фона = Белый} \\
	\bf И \> \tabfill{Форма внутреннего знака = Нет знака} \\
	\bf И \> \tabfill{Характер текста = Отсутствует} \\
	\bf И \> \tabfill{Вид транспорта = Частный (легковые, грузовые автомобили, мотоциклы, велосипеды)} \\
	\bf ТО \> \tabfill{Группа = Таблички к дорожным знакам}
	\end{tabbing}
	\item \begin{tabbing}
	\hspace{4em}\=\kill
	\bf ЕСЛИ \> \tabfill{Форма = Прямоугольник} \\
	\bf И \> \tabfill{Цвет фона = Белый} \\
	\bf И \> \tabfill{Форма внутреннего знака = Нет знака} \\
	\bf И \> \tabfill{Характер текста = Числа, елиницы измерения} \\
	\bf ТО \> \tabfill{Группа = Таблички к дорожным знакам}
	\end{tabbing}
	\item \begin{tabbing}
	\hspace{4em}\=\kill
	\bf ЕСЛИ \> \tabfill{Форма = Прямоугольник} \\
	\bf И \> \tabfill{Цвет фона = Белый} \\
	\bf И \> \tabfill{Форма внутреннего знака = Нет знака} \\
	\bf И \> \tabfill{Характер текста = Дата, время, дни недели} \\
	\bf ТО \> \tabfill{Группа = Таблички к дорожным знакам}
	\end{tabbing}
	\item \begin{tabbing}
	\hspace{4em}\=\kill
	\bf ЕСЛИ \> \tabfill{Форма = Прямоугольник} \\
	\bf И \> \tabfill{Цвет фона = Белый} \\
	\bf И \> \tabfill{Форма внутреннего знака = Нет знака} \\
	\bf И \> \tabfill{Характер текста = Георафические названия} \\
	\bf ТО \> \tabfill{Группа = Информационно-указательные знаки}
	\end{tabbing}
	\item \begin{tabbing}
	\hspace{4em}\=\kill
	\bf ЕСЛИ \> \tabfill{Форма = Прямоугольник} \\
	\bf И \> \tabfill{Цвет фона = Белый} \\
	\bf И \> \tabfill{Форма внутреннего знака = Нет знака} \\
	\bf И \> \tabfill{Характер текста = Слово СТОП} \\
	\bf ТО \> \tabfill{Группа = Информационно-указательные знаки}
	\end{tabbing}
	\item \begin{tabbing}
	\hspace{4em}\=\kill
	\bf ЕСЛИ \> \tabfill{Форма = Прямоугольник} \\
	\bf И \> \tabfill{Цвет фона = Белый} \\
	\bf И \> \tabfill{Форма внутреннего знака = Круг} \\
	\bf И \> \tabfill{Наличие стрелок = Нет} \\
	\bf ТО \> \tabfill{Группа = Запрещающие знаки}
	\end{tabbing}
	\item \begin{tabbing}
	\hspace{4em}\=\kill
	\bf ЕСЛИ \> \tabfill{Форма = Прямоугольник} \\
	\bf И \> \tabfill{Цвет фона = Белый} \\
	\bf И \> \tabfill{Форма внутреннего знака = Круг} \\
	\bf И \> \tabfill{Наличие стрелок = Есть} \\
	\bf ТО \> \tabfill{Группа = Предписывающие знаки}
	\end{tabbing}
	\item \begin{tabbing}
	\hspace{4em}\=\kill
	\bf ЕСЛИ \> \tabfill{Форма = Прямоугольник} \\
	\bf И \> \tabfill{Цвет фона = Белый} \\
	\bf И \> \tabfill{Форма внутреннего знака = Квадрат} \\
	\bf ТО \> \tabfill{Группа = Информационно-указательные знаки}
	\end{tabbing}
	\item \begin{tabbing}
	\hspace{4em}\=\kill
	\bf ЕСЛИ \> \tabfill{Форма = Прямоугольник} \\
	\bf И \> \tabfill{Цвет фона = Желтый} \\
	\bf ТО \> \tabfill{Группа = Информационно-указательные знаки}
	\end{tabbing}
	\item \begin{tabbing}
	\hspace{4em}\=\kill
	\bf ЕСЛИ \> \tabfill{Форма = Прямоугольник} \\
	\bf И \> \tabfill{Цвет фона = Синий} \\
	\bf И \> \tabfill{Вид транспорта = Отсутствует} \\
	\bf И \> \tabfill{Тип зданий = Отсутствуют} \\
	\bf ТО \> \tabfill{Группа = Знаки сервиса}
	\end{tabbing}
	\item \begin{tabbing}
	\hspace{4em}\=\kill
	\bf ЕСЛИ \> \tabfill{Форма = Прямоугольник} \\
	\bf И \> \tabfill{Цвет фона = Синий} \\
	\bf И \> \tabfill{Вид транспорта = Отсутствует} \\
	\bf И \> \tabfill{Тип зданий = Городские} \\
	\bf ТО \> \tabfill{Группа = Информационно-указательные знаки}
	\end{tabbing}
	\item \begin{tabbing}
	\hspace{4em}\=\kill
	\bf ЕСЛИ \> \tabfill{Форма = Прямоугольник} \\
	\bf И \> \tabfill{Цвет фона = Синий} \\
	\bf И \> \tabfill{Вид транспорта = Отсутствует} \\
	\bf И \> \tabfill{Тип зданий = Загородные} \\
	\bf ТО \> \tabfill{Группа = Знаки сервиса}
	\end{tabbing}
	\item \begin{tabbing}
	\hspace{4em}\=\kill
	\bf ЕСЛИ \> \tabfill{Форма = Прямоугольник} \\
	\bf И \> \tabfill{Цвет фона = Синий} \\
	\bf И \> \tabfill{Вид транспорта = Общественный (автобус, троллейбус, трамвай, самолет, поезд)} \\
	\bf ТО \> \tabfill{Группа = Информационно-указательные знаки}
	\end{tabbing}
	\item \begin{tabbing}
	\hspace{4em}\=\kill
	\bf ЕСЛИ \> \tabfill{Форма = Прямоугольник} \\
	\bf И \> \tabfill{Цвет фона = Синий} \\
	\bf И \> \tabfill{Вид транспорта = Частный (легковые, грузовые автомобили, мотоциклы, велосипеды)} \\
	\bf ТО \> \tabfill{Группа = Информационно-указательные знаки}
	\end{tabbing}
	\item \begin{tabbing}
	\hspace{4em}\=\kill
	\bf ЕСЛИ \> \tabfill{Форма = Прямоугольник} \\
	\bf И \> \tabfill{Цвет фона = Красный} \\
	\bf ТО \> \tabfill{Группа = Предупреждающие знаки}
	\end{tabbing}
	\item \begin{tabbing}
	\hspace{4em}\=\kill
	\bf ЕСЛИ \> \tabfill{Форма = Прямоугольник} \\
	\bf И \> \tabfill{Цвет фона = Зеленый} \\
	\bf ТО \> \tabfill{Группа = Информационно-указательные знаки}
	\end{tabbing}
	\item \begin{tabbing}
	\hspace{4em}\=\kill
	\bf ЕСЛИ \> \tabfill{Форма = Буква Х} \\
	\bf ТО \> \tabfill{Группа = Предупреждающие знаки}
	\end{tabbing}
	\item \begin{tabbing}
	\hspace{4em}\=\kill
	\bf ЕСЛИ \> \tabfill{Форма = Круг} \\
	\bf И \> \tabfill{Цвет фона = Белый} \\
	\bf И \> \tabfill{Наличие красной стрелки = Нет} \\
	\bf ТО \> \tabfill{Группа = Запрещающие знаки}
	\end{tabbing}
	\item \begin{tabbing}
	\hspace{4em}\=\kill
	\bf ЕСЛИ \> \tabfill{Форма = Круг} \\
	\bf И \> \tabfill{Цвет фона = Белый} \\
	\bf И \> \tabfill{Наличие красной стрелки = Есть} \\
	\bf ТО \> \tabfill{Группа = Знаки приоритета}
	\end{tabbing}
	\item \begin{tabbing}
	\hspace{4em}\=\kill
	\bf ЕСЛИ \> \tabfill{Форма = Круг} \\
	\bf И \> \tabfill{Цвет фона = Синий} \\
	\bf И \> \tabfill{Наличие окантовки = Нет} \\
	\bf ТО \> \tabfill{Группа = Предписывающие знаки}
	\end{tabbing}
	\item \begin{tabbing}
	\hspace{4em}\=\kill
	\bf ЕСЛИ \> \tabfill{Форма = Круг} \\
	\bf И \> \tabfill{Цвет фона = Синий} \\
	\bf И \> \tabfill{Наличие окантовки = Есть} \\
	\bf ТО \> \tabfill{Группа = Запрещающие знаки}
	\end{tabbing}
	\item \begin{tabbing}
	\hspace{4em}\=\kill
	\bf ЕСЛИ \> \tabfill{Форма = Круг} \\
	\bf И \> \tabfill{Цвет фона = Красный} \\
	\bf ТО \> \tabfill{Группа = Запрещающие знаки}
	\end{tabbing}
\end{enumerate}


Продукционные правила на языке Prolog:
\begin{minted}{prolog}
	roadsign("Предупреждающие знаки") :-
		priznak("Форма = Треугольник, вершина вверх"),!.
	roadsign("Знаки приоритета") :-
		priznak("Форма = Треугольник, вершина вниз"),!.
	roadsign("Знаки приоритета") :-
		priznak("Форма = Восьмиугольник"),!.
	roadsign("Знаки приоритета") :-
		priznak("Форма = Ромб"),!.
	roadsign("Информационно-указательные знаки") :-
		priznak("Форма = Кварат"),
		priznak("Наличие красной стрелки = Нет"),!.
	roadsign("Знаки приоритета") :-
		priznak("Форма = Кварат"),
		priznak("Наличие красной стрелки = Есть"),!.
	roadsign("Предупреждающие знаки") :-
		priznak("Форма = Прямоугольник"),
		priznak("Цвет фона = Белый"),
		priznak("Форма внутреннего знака = Нет знака"),
		priznak("Характер текста = Отсутствует"),
		priznak("Вид транспорта = Отсутствует"),
		priznak("Тип зданий = Отсутствуют"),!.
	roadsign("Информационно-указательные знаки") :-
		priznak("Форма = Прямоугольник"),
		priznak("Цвет фона = Белый"),
		priznak("Форма внутреннего знака = Нет знака"),
		priznak("Характер текста = Отсутствует"),
		priznak("Вид транспорта = Отсутствует"),
		priznak("Тип зданий = Городские"),!.
	roadsign("Таблички к дорожным знакам") :-
		priznak("Форма = Прямоугольник"),
		priznak("Цвет фона = Белый"),
		priznak("Форма внутреннего знака = Нет знака"),
		priznak("Характер текста = Отсутствует"),
		priznak("Вид транспорта = Общественный (автобус, троллейбус, трамвай, самолет, поезд)"),!.
	roadsign("Таблички к дорожным знакам") :-
		priznak("Форма = Прямоугольник"),
		priznak("Цвет фона = Белый"),
		priznak("Форма внутреннего знака = Нет знака"),
		priznak("Характер текста = Отсутствует"),
		priznak("Вид транспорта = Частный (легковые, грузовые автомобили, мотоциклы, велосипеды)"),!.
	roadsign("Таблички к дорожным знакам") :-
		priznak("Форма = Прямоугольник"),
		priznak("Цвет фона = Белый"),
		priznak("Форма внутреннего знака = Нет знака"),
		priznak("Характер текста = Числа, елиницы измерения"),!.
	roadsign("Таблички к дорожным знакам") :-
		priznak("Форма = Прямоугольник"),
		priznak("Цвет фона = Белый"),
		priznak("Форма внутреннего знака = Нет знака"),
		priznak("Характер текста = Дата, время, дни недели"),!.
	roadsign("Информационно-указательные знаки") :-
		priznak("Форма = Прямоугольник"),
		priznak("Цвет фона = Белый"),
		priznak("Форма внутреннего знака = Нет знака"),
		priznak("Характер текста = Георафические названия"),!.
	roadsign("Информационно-указательные знаки") :-
		priznak("Форма = Прямоугольник"),
		priznak("Цвет фона = Белый"),
		priznak("Форма внутреннего знака = Нет знака"),
		priznak("Характер текста = Слово СТОП"),!.
	roadsign("Запрещающие знаки") :-
		priznak("Форма = Прямоугольник"),
		priznak("Цвет фона = Белый"),
		priznak("Форма внутреннего знака = Круг"),
		priznak("Наличие стрелок = Нет"),!.
	roadsign("Предписывающие знаки") :-
		priznak("Форма = Прямоугольник"),
		priznak("Цвет фона = Белый"),
		priznak("Форма внутреннего знака = Круг"),
		priznak("Наличие стрелок = Есть"),!.
	roadsign("Информационно-указательные знаки") :-
		priznak("Форма = Прямоугольник"),
		priznak("Цвет фона = Белый"),
		priznak("Форма внутреннего знака = Квадрат"),!.
	roadsign("Информационно-указательные знаки") :-
		priznak("Форма = Прямоугольник"),
		priznak("Цвет фона = Желтый"),!.
	roadsign("Знаки сервиса") :-
		priznak("Форма = Прямоугольник"),
		priznak("Цвет фона = Синий"),
		priznak("Вид транспорта = Отсутствует"),
		priznak("Тип зданий = Отсутствуют"),!.
	roadsign("Информационно-указательные знаки") :-
		priznak("Форма = Прямоугольник"),
		priznak("Цвет фона = Синий"),
		priznak("Вид транспорта = Отсутствует"),
		priznak("Тип зданий = Городские"),!.
	roadsign("Знаки сервиса") :-
		priznak("Форма = Прямоугольник"),
		priznak("Цвет фона = Синий"),
		priznak("Вид транспорта = Отсутствует"),
		priznak("Тип зданий = Загородные"),!.
	roadsign("Информационно-указательные знаки") :-
		priznak("Форма = Прямоугольник"),
		priznak("Цвет фона = Синий"),
		priznak("Вид транспорта = Общественный (автобус, троллейбус, трамвай, самолет, поезд)"),!.
	roadsign("Информационно-указательные знаки") :-
		priznak("Форма = Прямоугольник"),
		priznak("Цвет фона = Синий"),
		priznak("Вид транспорта = Частный (легковые, грузовые автомобили, мотоциклы, велосипеды)"),!.
	roadsign("Предупреждающие знаки") :-
		priznak("Форма = Прямоугольник"),
		priznak("Цвет фона = Красный"),!.
	roadsign("Информационно-указательные знаки") :-
		priznak("Форма = Прямоугольник"),
		priznak("Цвет фона = Зеленый"),!.
	roadsign("Предупреждающие знаки") :-
		priznak("Форма = Буква Х"),!.
	roadsign("Запрещающие знаки") :-
		priznak("Форма = Круг"),
		priznak("Цвет фона = Белый"),
		priznak("Наличие красной стрелки = Нет"),!.
	roadsign("Знаки приоритета") :-
		priznak("Форма = Круг"),
		priznak("Цвет фона = Белый"),
		priznak("Наличие красной стрелки = Есть"),!.
	roadsign("Предписывающие знаки") :-
		priznak("Форма = Круг"),
		priznak("Цвет фона = Синий"),
		priznak("Наличие окантовки = Нет"),!.
	roadsign("Запрещающие знаки") :-
		priznak("Форма = Круг"),
		priznak("Цвет фона = Синий"),
		priznak("Наличие окантовки = Есть"),!.
	roadsign("Запрещающие знаки") :-
		priznak("Форма = Круг"),
		priznak("Цвет фона = Красный"),!.
\end{minted}
