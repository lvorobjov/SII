\chapter{Ход работы}

\textbf{Цель работы}: познакомиться с функциями инженера по знаниям на этапе
построения концептуальной модели предметной области; освоить методику
выявления и структурирования знаний о способах решения задач в заданной
предметной области.

\textbf{Вариант 18}: Система обучения управлению автомобилем и правил дорожного
движения.

\section{Оценка «экспертности» предметной области}

\begin{table}[p]
\caption{Оценка «экспертности» предметной области}
\begin{tabularx}{\linewidth}{|X|c|c|} \hline
	\multicolumn{1}{|c|}{Вопрос} &
	\multicolumn{1}{c|}{Да} &
	\multicolumn{1}{c|}{Нет} \\ \hline
	
	1.Сильно ли помогает опыт специалисту при решении задачи?	& + & \\ \hline
	2.Велика ли разница во времени и качестве решения у новичка и специалиста?	& + & \\ \hline
	3.Имеются ли эксперты, готовые поделится своим опытом?	& + & \\ \hline
	4.Часто ли возникает потребность в решении задачи?	& + & \\ \hline
	5.Может ли быть точно очерчена предметная область?	& 	& + \\ \hline
	6.Требуется ли знание эвристик?	& 	& + \\ \hline
	7.Решение не требует большого количества вычислений?	& + & \\ \hline
	8.Есть ли «шум» во входных данных (нечёткость, неполнота, некорректность)?	& + & \\ \hline
	9.Имеются ли большое количество объектов, признаков объектов и связей между ними?	& + & \\ \hline
	10.Имеются ли сомнения в достоверности информации?	& 	& + \\ \hline
	11.Есть ли необходимость в принятии решения с определением степени уверенности в этом решении?	& + & \\ \hline
	12.Занимает ли решение задачи значительное время?	& + & \\ \hline
	13.Являются ли традиционные математические модели и ранее разработанные пакеты программ непригодными для получения решения?	& 	& + \\ \hline
	14.Согласны ли потенциальные пользователи использовать экспертную систему?	& + & \\ \hline
	15.Имеется ли доступная техника и ПО для реализации будущей ЭС?	& + & \\ \hline
	16.Достаточна ли квалификация имеющихся специалистов для разработки ЭС?	& + & \\ \hline
\end{tabularx}
\end{table}

Экспертность системы:
$$ Q = \frac{12}{16} \cdot 100\% = 75 \% $$

$Q > 50\%$, следовательно предметная область «экспертна», т.е. она восприимчива к внедрению ЭС.

\section{Формулировка задач, решаемых по технологии ЭС}

ЭС выполняет следующие задачи:
\begin{itemize}
	\item формулирует вопросы и предлагает варианты правильных ответов по теме безопасности дорожного движения;
	\item диагностирует ошибки студента, и подсказывает правильные решения на примерах;
	\item аккумулирует знания об «ученике», его характерных ошибках и подбирает задания для их ликвидации.
\end{itemize}

\section{Назначение проектируемой экспертной системы и её потенциальные	пользователи}

ЭС предназначена для обучения водителей правилам, которые:
\begin{itemize}
	\item устанавливают порядок проезда перекрестков;
	\item ограничивают максимальную скорость движения;
	\item разрешает или запрещает обгон, остановку, стоянку, движение задним ходом, разворот;
	\item помогают избегать ДТП на дороге по рекомендациям БДД;
	\item разрешают или запрещают буксировку транспортных средств в соответствии с ПДД;
	\item регламентируют необходимость выключения внешних световых приборов транспортных средств.
\end{itemize}

В данном случае ЭС предназначена для обучения водителей правилам, которые помогают избегать ДТП на дороге.

Студенту предлагается ситуация, в которой необходимо выбрать правильное действие. Если обучающийся выбрал неправильное действие, система предлагает другую похожую ситуацию, в которой необходимо выбрать правильное действие. Если два раза ошибка, то ЭС показывает правило, которым нужно было руководствоваться.

\section{Концептуальное описание предметной области}

За основу взят фрагмент из книги Э.С.Циганков «Управление автомобилем в критической ситуации». Глава «Стабилизация автомобиля при потере устойчивости и управляемости».

Снос, занос, вращение, опрокидывание в различных сочетаниях требуют определенных действий.

% \textbf{Стабилизация при заносе} --- при возникновении бокового скольжения задней оси, что чаще всего возникает на автомобиле с задними ведущими колесами (классическая компоновка), компенсировать занос можно быстрым поворотом рулевого колеса в сторону заноса на 90-180° без смены положения рук. Если амплитуда заноса большая, водитель переходит к поочередному рулению левой и правой руками со сменой положения рук на боковом секторе.

\subsection{Снос}

\textbf{Снос – боковое скольжение передних колес} – чаще всего возникает при экстренных маневрах и прохождении поворота на критической скорости. 

Снос передних колес следует рассматривать как результат грубой ошибки водителя, так как он всегда сопровождается частичной потерей управляемости. Сигналом о совершенной ошибке служит “визг” передних покрышек на сухом покрытии.

Прекратить или уменьшить снос передних колес можно двумя способами: либо увеличить загрузку передних колес, либо уменьшить угол их поворота, чтобы от скольжения перейти к качению.

Можно рекомендовать несколько приемов безопасности при сносе.

1. Торможение двигателем на постоянной передаче.

2. Торможение двигателем на понижающей передаче.

3. Легкое подтормаживание левой ногой для увеличения загрузки переднего наружного колеса. Режим торможения плавный, с постоянным тормозным усилием, исключающим блокирование колес.

4. Выравнивание управляемых колес (если это позволяют ситуация и ширина проезжей части).

5. Выравнивание и повторный вход с загрузкой передних колес.Чтобы вернуть автомобилю управляемость, потерянную при сносе передних колес в повороте: выполните повторный вход (вначале выровняйте колеса, а затем вновь поверните их); загрузите наружное переднее колесо любым доступным вам способом (торможением двигателем, легким подтор-маживанием левой ногой, включением понижающей передачи). Преодолейте страх и откажитесь от резкого торможения.

\subsection{Занос малой амплитуды} 

Не ждите усиления заноса, реагируйте сразу резким рывком обеих рук в сторону заноса. Одновременно с поворотом рулевого колеса “прикройте газ”. Если вы тотчас мягким движением не выровняете колеса, может возникнуть ритмический занос, направленный в противоположенную сторону.

\subsection{Глубокий занос}

Если вам не удалось стабилизировать автомобиль на ранней стадии заноса, то после рывка двумя руками выполните “доворот” одной из рук, которая окажется в верхнем секторе рулевого колеса; одновременно “прикройте газ”. Лучше выполнять “доворот” левой рукой, чтобы освободить правую для экстренного включения понижающей передачи. Этот прием поможет повысить тягу двигателя, чтобы преодолеть центробежную силу, выбрасывающую автомобиль с дороги.

\subsection{Критический занос}

Преодолеть критический занос можно тремя путями: выровнять автомобильза счет сверхскоростного руления на пределе возможностей подготовленного водителя; развернуть автомобиль на 360° относительно прямого направления, используя для этого серию приемов высшего мастерства (см. прием 40); использовать сильный тормозной эффект для снижения скорости, сохраняя угол заноса коррегирующим рулением и переменным дросселированием.

\subsection{Ритмический занос}

В основе критической ситуации ритмического заноса лежит явление “динамический хлыст”, когда раскачиванию задней оси автомобиля в поперечном направлении сопутствует резонанс, из-за чего амплитуда каждого последующего заноса возрастает до критического значения. После этого начинается неуправляемое вращение автомобиля.

Преодолеть серию импульсов ритмического заноса можно серией противонаправленных рывков рулевого колеса без смены или со сменой хвата рук при больших углах заноса. Желательно избегать задержек при сменах направления руления и каждым последующим импульсом стремиться опередить развитие заноса.

\subsection{Вращение вокруг задней оси на 180°}

Возникновению вращения предшествует обычно грубая ошибка в управлении (резкое торможение с длительным блокированием колес, опоздание с реакцией на глубокий занос, замедленное руление, резкое дросселирование и др.).

Возникновению вращения предшествует обычно грубая ошибка в управлении (резкое торможение с длительным блокированием колес, опоздание с реакцией на глубокий занос, замедленное руление, резкое дросселирование и др.).

Выключите сцепление и предельно проверните рулевое колесо навстречу вращению. Если до момента разворота вы успели повернуть его в сторону заноса, то можете ограничиться только выключением сцепления.

Включите сцепление и выровняйте колеса до полного разворота на 360°.

\subsection{Вращение на 360°}

Если вы не знаете, что делать во время вращения, не делайте ничего и не тормозите. Автомобиль сам стабилизируется.

Постарайтесь перевести неуправляемое вращение в управляемое. Для этого в первой фазе увеличьте частоту вращения коленчатого вала двигателя до максимальной и поверните рулевое колесо в сторону разворота. Во второй фазе (после вращения на 180°) выключите сцепление, быстро поверните рулевое колесо в противоположную сторону до упора и выполните “полицейский разворот” (см. прием 39).

Опасайтесь наружного “упора”, так как удар о препятствие может привести к опрокидыванию.

\subsection{Силовое руление при повреждении передней подвески}

Наиболее опасны ситуации, связанные с повреждением передней подвески или колеса (разрывом рулевой тяги, ударом передним колесом о бордюр, повреждением передней шины). Возникающее вращение вокруг поврежденного колеса имеет высокую интенсивность и, самое неприятное, дефект не дает возможности своевременно стабилизировать автомобиль.

Удержать на дороге автомобиль с поврежденной передней подвеской или колесом вы сможете, если приложите максимальное усилие обеих рук одновременно.

Сожмите кисти на рулевом колесе, напрягите мышцы рук и спины, препятствуя вращению рулевого колеса.

Перевод автомобиля на “упор после скольжения”

Если автомобиль выносит наружу на обледенелом повороте, используйте снежное препятствие как опору для торможения и возврата управляемости.

На автомобиле с классической компоновкой переходите на “упор” задним наружным колесом, на переднеприводном – передним. Смягчить удар о препятствие можно за счет пробуксовки ведущего колеса.

\subsection{Стабилизация при боковом опрокидывании}

Аварийная ситуация, связанная с опрокидыванием автомобиля, является следствием критических ситуаций: сноса передней оси, бокового скольжения, критического и ритмического заносов, вращения автомобиля. Сами по себе эти критические ситуации не могут перерасти в опрокидывание до тех пор, пока скольжение не прерывается “упором” – боковым ударом о препятствие (яму, бугор, выступ, бордюр и др.).

Страхуясь от опрокидывания, прекратите торможение, с силой поверните рулевое колесо в сторону опрокидывания, а затем выровняйте автомобиль.

Общие рекомендации:
\begin{itemize}
	\item не тормозить!
	\item повернуть рулевое колесо в сторону опрокидывания, а затем выровнять;
	\item продолжать при необходимости стабилизирующие действия способами силового или скоростного руления и переменного дросселирования, чтобы преодолеть последствия критической ситуации.
\end{itemize}

\section{Дерево решений}

\begin{figure}[H]
	\centering
	\includegraphics[width=1\linewidth]{fig/tree1}
	\caption{Дерево решений задачи выбора действия в критической ситуации}
	\label{fig:tree1}
\end{figure}

\section{Дерево утверждений и фактов}

\begin{figure}[h]
	\centering
	\includegraphics[width=1\linewidth]{fig/tree2}
	\caption{Дерево утверждений и фактов для определения вида критической ситуации}
	\label{fig:tree2}
\end{figure}

\section{Продукционные правила}

На основании дерева утверждений и фактов получаются следующие продукционные правила:
\begin{itemize}
	\item \textbf{ЕСЛИ} Не все колеса касаются земли \textbf{ТО} устранить опрокидывание: силовое руление вправо (влево), балансирование, выравнивание;
	\item \textbf{ЕСЛИ} Все колеса касаются замли \textbf{И} Передняя ось скользит вбок \textbf{ТО} устранить снос передней оси: выравнивание рулевым колесом, торможение двигателем, подтормаживание левой ногой;
	\item \textbf{ЕСЛИ} Все колеса касаются замли \textbf{И} Передняя ось не скользит вбок \textbf{И} Задняя ось скользит вбок \textbf{И} Есть колебательные движения задней оси \textbf{ТО} устранить ритмический занос серией скоростных импульсов руления в одну и другую сторону;
	\item \textbf{ЕСЛИ} Все колеса касаются замли \textbf{И} Передняя ось не скользит вбок \textbf{И} Задняя ось скользит вбок \textbf{И} Нет колебательных движений задней оси \textbf{И} Амплитуда поворота задней оси меньше максимальной амплитуды руля \textbf{ТО} Устранить занос резким поворотом рулевого колеса в сторону заноса до направления прямолинейного движения, мягкое выравнивание;
	\item \textbf{ЕСЛИ} Все колеса касаются замли \textbf{И} Передняя ось не скользит вбок \textbf{И} Задняя ось скользит вбок \textbf{И} Нет колебательных движений задней оси \textbf{И} Амплитуда поворота задней оси больше максимальной амплитуды руля \textbf{ТО} Устранить критический занос скоростным рулением полной амплитуды;
	\item \textbf{ЕСЛИ} Все колеса касаются замли \textbf{И} Передняя ось не скользит вбок \textbf{И} Задняя ось не скользит вбок \textbf{ТО} Продолжать управление автомобилем, потери устойчивости нет.
\end{itemize}